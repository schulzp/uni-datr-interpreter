\documentclass[a4paper]{sprach}

\usepackage{fontspec}% Schriftumschaltung mit den nativen XeTeX-Anweisungen
                     % vornehmen. Voreinstellung: Latin Modern

% Hurenkinder und Schusterjungen verhindern
\clubpenalty10000
\widowpenalty10000
\displaywidowpenalty=10000

\usepackage{fontspec,xltxtra,xunicode,xspace,microtype,csquotes}
\defaultfontfeatures{Ligatures=TeX}

\setromanfont[Ligatures={Common,TeX}]{Minion Pro}
\setsansfont[Scale=MatchLowercase]{Myriad Pro}
\usepackage{polyglossia}% Sprachumschaltung

\usepackage{pdflscape}

\usepackage{caption,subcaption}

\usepackage{verbatim}
\usepackage{amsmath}
\usepackage{mathtools}
\usepackage{unicode-math}
\setmathfont{XITS Math}

\usepackage{avm}
\usepackage{tikz}
\usepackage{tikz-qtree}

\usepackage{footnote}
\makesavenoteenv{tabular}
\makesavenoteenv{table}

\DeclarePairedDelimiterX\typeinstance[1]{⟪}{⟫}{#1}
\newcommand{\lmd}[1]{\lambda #1}
\newcommand{\type}[1]{\langle #1 \rangle}
\newcommand{\cat}[1]{\text{#1}}
\newcommand{\func}[2]{\text{#1}(#2)}
\newcommand{\kgr}[1]{\langle #1 \rangle}

\setdefaultlanguage{german}% Voreingestellte Dokumentsprache: Deutsch

\begin{document}

\Abgabeblatt{2}{10. Januar 2017}{Hagen Langer}{Lea Jakubigk}{Peter Schulz}{}

\tableofcontents

\section*{Aufgabe 1}


\begin{table}[h]
    \centering
    \begin{tabular}{llllll}
    \hline
     ID & folgt aus (ID) & von i & bis j & links & $\rightarrow$ rechts     \\ \hline
    1   & 0              & 0     & 0     & S     & * S KONJ S \\
    2   & 0|1            & 0     & 0     & S     & * NP VP    \\
    3   & 2              & 0     & 0     & NP    & * NP PP    \\
    4   & 2|3            & 0     & 0     & NP    & * DET N    \\
    4   & 2|3            & 0     & 1     & NP    & DET * N    \\
    4   & 2|3            & 0     & 2     & NP    & DET N *    \\
    2   & 0              & 0     & 2     & S     & NP * VP    \\
    3   & 2              & 0     & 2     & NP    & NP * PP    \\
    5   & 3              & 2     & 2     & PP    & * P NP \textsuperscript{(1)}\\
    6   & 2              & 2     & 2     & VP    & * V NP PP  \\
    7   & 2              & 2     & 2     & VP    & * V NP     \\
    6   & 2              & 2     & 3     & VP    & V * NP PP  \\
    7   & 2              & 2     & 3     & VP    & V * NP     \\
    8   & 6|7            & 3     & 3     & NP    & * NP PP    \\
    9   & 6|7|8          & 3     & 3     & NP    & * DET N    \\
    9   & 6|7|8          & 3     & 4     & NP    & DET * N    \\
    9   & 6|7|8          & 3     & 5     & NP    & DET N *    \\
    8   & 6|7            & 3     & 5     & NP    & NP * PP    \\
    6   & 2              & 2     & 5     & VP    & V NP * PP  \\
    7   & 2              & 2     & 5     & VP    & V NP *     \\
    10  & 6|8            & 5     & 5     & PP    & * P NP     \\
    10  & 6|8            & 5     & 6     & PP    & P * NP     \\
    11  & 10             & 6     & 6     & NP    & * NP PP    \\
    12  & 10|11          & 6     & 6     & NP    & * DET N    \\
    12  & 10|11          & 6     & 7     & NP    & DET * N    \\
    12  & 10|11          & 6     & 8     & NP    & DET N *    \\
    11  & 10             & 6     & 8     & NP    & NP * PP \textsuperscript{(2)}\\
    10  & 6              & 5     & 8     & PP    & P NP *     \\
    6   & 2              & 2     & 8     & VP    & V NP PP *  \\
    2   & 0|1            & 0     & 8     & S     & NP VP *    \\
    1   & 0              & 0     & 8     & S     & S * KONJ S \textsuperscript{(2)}\\
    \end{tabular}
    \caption{Earley-Parser-Zustände für die Eingabe:
\textit{\textsubscript{0} 
der \textsubscript{1}
Hund \textsubscript{2}
sieht \textsubscript{3}
die \textsubscript{4}
Katze \textsubscript{5}
mit \textsubscript{6}
dem \textsubscript{7}
Fernrohr\textsubscript{8}}.
      \textsuperscript{(1)}Kein P an (2,2)$\rightarrow$ Nicht weiter verfolgen,
      \textsuperscript{(2)}Ende erreicht $\rightarrow$ Nicht weiter verfolgen
    }
\end{table}


\section*{Aufgabe 2}
\subsection*{Lexikoneinträge}
\subsubsection*{Nomen}

\begin{avm}
    Hund \[
    CAT & N\\
    AGR & \[
     		KASUS & nom $\vee$ akk $\vee$  dat  \\
    		NUM & sg \\
    		PER & 3 \\
    		GEN & mask   		
    	\]
    \]
\end{avm},

\begin{avm}
    Katze \[
    CAT & N\\
    AGR & \[
   			KASUS & nom $\vee$ akk $\vee$  dat  \\
    		NUM & sg \\
    		PER & 3 \\
    		GEN & fem 
    	\]
    \]
\end{avm}

\subsubsection*{Determinierer}

\begin{avm}
der \[
CAT & DET\\
AGR & \[
		KASUS & nom \\
    	NUM & sg \\
    	GEN & mask 	
	\] $\vee$ \[
		KASUS & dat \\
    	NUM & sg \\
    	GEN & fem 
    \]
\]
\end{avm},

\begin{avm}
die \[
CAT & DET\\
AGR & \[
		KASUS & nom $\vee$ akk \\
    	NUM & sg \\
    	GEN & fem 
	\]
\]
\end{avm},

\begin{avm}
dem \[
CAT & DET\\
AGR & \[
    	KASUS &  dat  \\
    	NUM & sg \\
    	GEN & mask 
	\]
\]
\end{avm},

\begin{avm}
den \[
CAT & DET\\
AGR & \[
	    KASUS & akk  \\
    	NUM & sg \\
    	GEN & mask 
	\]
\]
\end{avm}

\subsubsection*{Verben}

\begin{avm}
sieht \[
CAT & V\\
AGR & \[
    	NUM & sg \\
    	PER & 3 
	\]\\
SUBCAT & \[
		ARG$_{\text{NP}_\text{Subj}}$ & \[
        	CAT & NP\\
            AGR & \[
            	KASUS & nom
            \]
        \]\\
        ARG$_{\text{NP}_\text{Obj}}$ & \[
         	CAT & NP\\
            AGR & \[
            	KASUS & akk
            \]       
        \]
	\]
\]
\end{avm},

\begin{avm}
schläft \[
CAT & V\\
AGR & \[
    	NUM & sg \\
    	PER & 3 
	\]\\
SUBCAT & none
\]
\end{avm}

\subsubsection*{Präposition}

\begin{avm}
mit \[
CAT & P\\
SUBCAT & \[
		ARG$_\text{NP}$ & \[
        	CAT & NP\\
            AGR & \[
            	KASUS & dat
            \]
        \]
	\]
\]
\end{avm}

\subsection*{Definierte Kategorien}

\begin{avm}
S \[
CAT & S\\
SUBCAT & \[
		ARG$_\text{NP}$ & \[
        	CAT & NP
        \]\\
        ARG$_\text{VP}$ & \[
        	CAT & VP
        \]
	\]
\]
\end{avm}


\begin{avm}
NP \[
CAT & NP\\
SUBCAT & \[
		ARG$_\text{DET}$ & \[
        	CAT & DET
        \]\\
        ARG$_\text{N}$ & \[
        	CAT & N
        \]
	\] $\vee$ \[
    	ARG$_\text{NP}$ & \[
        	CAT & NP
        \]\\
        ARG$_\text{PP}$ & \[
        	CAT & PP
        \]
    \]
\]
\end{avm}

\begin{avm}
VP \[
CAT & VP\\
SUBCAT & \[
		ARG$_\text{V}$ & \[
        	CAT & V
        \]\\
        ARG$_\text{NP}$ & \[
        	CAT & NP
        \]
	\]
\]
\end{avm}

\begin{avm}
PP \[
CAT & PP\\
SUBCAT & \[
		ARG$_\text{P}$ & \[
        	CAT & P
        \]\\
        ARG$_\text{NP}$ & \[
        	CAT & NP
        \]
	\]
\]
\end{avm}

\subsubsection*{Ergänzende Regeln}

Die folgenden Regeln ergänzen die zuvor aufgelisteten Einträge. Es wird eine vereinfachte/verkürzte Notation
verwendet: Im Kontext von $\cat{S} \rightarrow \cat{NP}\;\cat{VP}$ steht
$\kgr{\text{VP|\dots}}$ (Gleichung \ref{eq:akk}) wobei VP $\equiv$ S|SUBCAT|ARG$_\text{VP}$.

\begin{alignat}{2}
\cat{S} &\rightarrow \cat{NP}\;\cat{VP}\\ 
  &\; \kgr{\text{NP|SUBCAT|ARG}_\text{N}\text{|KASUS}} = nom\label{eq:nom}\\
  &\; \kgr{\text{VP|SUBCAT|ARG}_{\text{NP}_\text{Obj}}\text{|SUBCAT|ARG}_\text{N}|\text{KASUS}} = akk\label{eq:akk}\\
\cat{NP} &\rightarrow \cat{DET}\;\cat{N}\\
  &\; \kgr{\text{DET|AGR}} = \kgr{\text{N|AGR}}\label{eq:genus}
%\cat{NP} &\rightarrow \cat{NP}\;\cat{PP}\\
%\cat{VP} &\rightarrow \cat{V}\;\cat{NP}\\
%\cat{PP} &\rightarrow \cat{P}\;\cat{NP}\\
%  &\;\kgr{\text{P|SUBCAT|ARG}_\text{P}|\text{AGR|KASUS}} = \kgr{\text{NP|SUBCAT|ARG}_\text{NP}\text{|SUBCAT|ARG}_\text{N}|\text{KASUS}}
\end{alignat}

Dadurch werden folgende ungrammatikalischen Ketten als solche erkannt:

\begin{description}
  \item [\textit{das Hund} sieht die Katze] wird durch \ref{eq:genus} ausgeschlossen.
  \item [\textit{den Hund} sieht den Hund] wird durch \ref{eq:nom} ausgeschlossen.
  \item [die Katze \textit{schläft} den Hund] wird durch \ref{eq:akk} ausgeschlossen.
\end{description}


\section*{Aufgabe 3}

\begin{figure}[ht]
  \centering
  \begin{subfigure}[b]{0.4\linewidth}
    \begin{tikzpicture}
      \tikzset{every tree node/.style={align=center,anchor=north}}
      \tikzset{level distance=30pt}
      \Tree [.S 
        [.NP [.DET der ] [.N Hund ] ]
        [.VP [.V sieht ] [.NP [.DET die ] [.NP [.ADJ braune ] [.N Katze ] ] ] ] ]
    \end{tikzpicture}
    \caption{Syntaxbaum}
  \end{subfigure}
  \begin{subfigure}[b]{0.4\linewidth}
    \begin{align*}
      \cat{S} &\rightarrow \cat{NP}\;\cat{VP}\\
      \cat{NP} &\rightarrow \cat{DET}\;\cat{N}\\
      \cat{NP} &\rightarrow \cat{ADJ}\;\cat{N}\\
      \cat{VP} &\rightarrow \cat{V}\;\cat{NP}\\\cline{1-2}
      \cat{V} &\rightarrow \text{sieht}\\
      \cat{N} &\rightarrow \text{Hund}\;|\;\text{Katze}\\
      \cat{ADJ} &\rightarrow \text{braune}\\
      \cat{DET} &\rightarrow \text{der}\;|\;\text{die}
    \end{align*}
    \caption{Kategorialgrammatik}
  \end{subfigure}
  \caption{Zerlegung}
\end{figure}

\begin{landscape}
  \begin{figure}[p]
\begin{tikzpicture}
\tikzset{every tree node/.style={align=center,anchor=north}}
\Tree [.\node (n:satz) {$t$};
  [.{$\func{sehen}{\text{dbk}}(\text{dh})$}
    [.\node (n:der_hund) {$e$}; 
      [.{$\exists x\Big(\func{h}{x}\wedge \forall y\big(\func{h}{y} \rightarrow x = y\big)\Big) := \text{dh}$}
        [.\node (n:der) {$\type{\type{e,t},e}$}; 
          [.{$\lmd{P}.\exists x\big(P(x)\wedge \forall y\big(P(y) \rightarrow x = y\big)\big)$} {der} ]
        ]
        [.\node (n:hund) {$\type{e,t}$}; 
          [.{$\lmd{x}.\func{h}{x}$} {Hund} ]
        ] 
      ]
    ]
    [.\node (n:x_sieht_die_braune_katze) {$\type{e,t}$};
      [.{$\lmd{x}.\func{sehen}{\text{dbk}}(x)$} 
        [.\node (n:sieht) {$\type{e,\type{e,t}}$}; 
          [.{$\lmd{y}\lmd{x}.\func{sehen}{y}(x)$} {sieht} ] 
        ]
        [.\node (n:die_braune_katze) {$e$};
          [.{$\exists x\Big(\func{bk}{x}\wedge \forall y\big(\func{bk}{y} \rightarrow x = y\big)\Big) := \text{dbk}$} 
            [.\node (n:die) {$\type{\type{e,t},e}$}; 
              [.{$\lmd{P}.\exists x\big(P(x)\wedge \forall y\big(P(y) \rightarrow x = y\big)\big)$} {die} ]
            ]
            [.\node (n:braune_katze) {$\type{e,t}$};
              [.{$\lmd{x}.\big(\func{b}{x} \wedge \func{k}{x}\big) := \lmd{x}.\func{bk}{x}$} 
                [.\node (n:braune) {$\type{e,t}$}; [.{$\lmd{x}.\func{b}{x}$} {braune} ] ]
                [.\node (n:katze) {$\type{e,t}$}; [.{$\lmd{x}.\func{k}{x}$} {Katze} ] ] 
              ]  
            ]
          ]
        ]
      ]
    ] 
  ]
]
\end{tikzpicture}
\begin{equation*}
  \typeinstance{\text{der Hund sieht die braune Katze}} = \exists x\exists y
  \Big[
    \Big(
      \func{h}{x}
      \wedge
      \forall z \big( \func{h}{z} \rightarrow x = z \big)
    \Big)
    \wedge
    \Big(
      \big( \func{b}{y} \wedge \func{k}{y} \big)
      \wedge
      \forall z \big( \func{b}{z} \wedge \func{k}{z} \rightarrow y = z \big)
    \Big)
    \wedge
    \text{sehen}(y)(x)
  \Big]
\end{equation*}

\caption{Typen und logische Entsprechung}
\end{figure}


\end{landscape}


\end{document}
